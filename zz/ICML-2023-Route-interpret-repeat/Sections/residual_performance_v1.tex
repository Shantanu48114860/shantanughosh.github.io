%  No mention of coparison of baselines
\cref{fig:expert_performance_cv_vit} (a-c) display the proportional accuracy of the experts and the residuals of our method per iteration. The proportional accuracy of each model (experts and/or residuals) is defined as the accuracy of that model times its coverage. Recall that the model's coverage is the empirical mean of the samples selected by the selector. 
% Since some experts only cover a ``Benign'' or a ``Malignant'' sample, we provide the accuracy rather than the AUROC for HAM10000 herewith because it would otherwise result in a random AUROC for that expert.
\cref{fig:expert_performance_cv_vit}a show that the experts and residual cumulatively achieve an accuracy $\sim$ 0.92 for the CUB-200 dataset in iteration 1, with more contribution from the residual (black bar) than the expert1 (blue bar). Later iterations cumulatively increase and worsen the performance of the experts and corresponding residuals, respectively. The final iteration carves out the entire interpretable portion from the Blackbox $f^0$ via all the experts, resulting in their more significant contribution to the cumulative performance. The residual of the last iteration covers the ``hardest'' samples, achieving low accuracy. Tracing these samples back to the original Blackbox $f^0$, it also classifies these samples poorly (\cref{fig:expert_performance_cv_vit}{(d-f)}).
As shown in the coverage plot, this experiment reinforces~\cref{fig:Schematic}, where the flow through the experts gradually becomes thicker compared to the narrower flow of the residual with every iteration. Refer to~\cref{fig:expert_performance_cv_resnet} in the~\cref{app:resnet_cv} for the results of the ResNet-based MoIEs.


